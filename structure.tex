%----------------------------------------------------------------------------------------
%	REQUIRED PACKAGES AND MISC CONFIGURATIONS
%----------------------------------------------------------------------------------------

% \usepackage{tfrupee} % Required for ruppe symbol but need to install sudo apt-get install texlive-fonts-extra (1,023MB) & use \rupee

\usepackage{graphicx} % Required for including images

\usepackage[
	a4paper, % Change to letterpaper for US Letter
	top=2.5cm, 
	bottom=2.5cm, 
	left=2.5cm, 
	right=2.5cm
]{geometry} % Document margins

\usepackage{fp} % Required for invoice calculations

\usepackage[group-separator={,},group-minimum-digits=4]{siunitx} % Required for automatically adding commas to large numbers and delimitting for 4 digits (e.g. 1,000)

\usepackage{advdate} % Required for date calculation

%----------------------------------------------------------------------------------------
%	FONTS
%----------------------------------------------------------------------------------------

\usepackage[utf8]{inputenc} % Required for inputting international characters
\usepackage[T1]{fontenc} % Output font encoding for international characters

\usepackage{tgadventor} % Use the TeX Gyre Adventor font
\renewcommand*\familydefault{\sfdefault} % Set the base font of the document to sans serif

%----------------------------------------------------------------------------------------
%	COLOURS
%----------------------------------------------------------------------------------------

\usepackage{xcolor} % Required for defining and using custom colours

\definecolor{highlightcolour}{HTML}{DF0174} % Colour used for making text stand out
\definecolor{rulecolour}{HTML}{B2BEB5} % Colour used for rules

%----------------------------------------------------------------------------------------
%	TABLES
%----------------------------------------------------------------------------------------

\usepackage{colortbl} % Required for colouring table cells (used for rules)

\usepackage{booktabs} % Required for nicer table rules

\usepackage{multirow} % Required for allowing cells to take up multiple rows in tables

\usepackage{array} % Required for customizing table spacing and features
\def\arraystretch{1.2} % Table row spacing, 1 is the default

\newcolumntype{R}[1]{>{\raggedleft\let\newline\\\arraybackslash\hspace{0pt}}m{#1}} % Define a fixed-width right-aligned column type

%----------------------------------------------------------------------------------------
%	CUSTOM COMMANDS
%----------------------------------------------------------------------------------------

\newcommand{\payeename}[1]{\renewcommand{\payeename}{#1}}
\newcommand{\payeejob}[1]{\renewcommand{\payeejob}{#1}}
\newcommand{\payeeaddresslineone}[1]{\renewcommand{\payeeaddresslineone}{#1}}
\newcommand{\payeeaddresslinetwo}[1]{\renewcommand{\payeeaddresslinetwo}{#1}}
\newcommand{\payeecontactlineone}[1]{\renewcommand{\payeecontactlineone}{#1}}
\newcommand{\payeecontactlinetwo}[1]{\renewcommand{\payeecontactlinetwo}{#1}}

\newcommand{\billdate}[1]{\renewcommand{\billdate}{#1}}
\newcommand{\billcycle}[1]{\renewcommand{\billcycle}{#1}}
\newcommand{\fromdate}[1]{\renewcommand{\fromdate}{#1}}
\newcommand{\todate}[1]{\renewcommand{\todate}{#1}}

%------------------------------------------------

% Running variables
\gdef\variableA{0}
\gdef\variableB{0}
\gdef\variableC{0}
\gdef\variableD{0}

% Cumulative variables
\gdef\TotalA{0} % Total before tax
\gdef\TotalB{0} % Tax
\gdef\TotalC{0} % Net after tax

\newcommand{\taxrate}[1]{\renewcommand{\taxrate}{#1}} % Tax rate used to automatically calculate tax

%----------------------------------------------------------------------------------------
%	INVOICE TABLES
%----------------------------------------------------------------------------------------

% Payee information (top table)
\newcommand{\invoicedtotable}{
	{
	\centering
	\footnotesize
	\begin{tabular}{p{0.30\textwidth} | p{0.30\textwidth} | p{0.30\textwidth}}
		\arrayrulecolor{rulecolour}\toprule[0.5pt] % Horizontal line at the top of the table
		% \multirow{3}{*}{{\color{highlightcolour} \Huge INVOICE}} & \textbf{Recipient} & \\
		\normalsize\payeename & \normalsize\payeeaddresslineone & \normalsize\payeecontactlineone \\ 
		\vspace{0.5cm}\\
		\normalsize\payeejob & \normalsize\payeeaddresslinetwo & \normalsize\payeecontactlinetwo \\
		\arrayrulecolor{rulecolour}\bottomrule[0.5pt] % Horizontal line at the bottom of the table
	\end{tabular}
	}
}


%-------------------------------------------------------------------------------------
%            Invoice items table
%---------------------------------------------------------------------------------------
\newenvironment{invoicetable}
{
	\begin{tabular}{p{0.43\textwidth} R{0.12\textwidth} R{0.12\textwidth} R{0.12\textwidth} R{0.12\textwidth}}
			\arrayrulecolor{rulecolour}\toprule[0.5pt]
		\textbf{Source of Power} & \textbf{Grid} & \textbf{Diesel} & \textbf{Total} & \textbf{Units} \\
		\arrayrulecolor{rulecolour}\toprule[0.5pt]
}
{	
		\arrayrulecolor{rulecolour}\bottomrule[0.5pt]
		\\ [-1em] % Reduce whitespace before the totals
		\textbf{Bill Payable (Rounded)} & \num{670} & \num{20} & \num{690} &{Rs}\\ 
	\end{tabular}
}


%----------------------------------------------------------------------------------------
%	INVOICE ITEM LINES
%----------------------------------------------------------------------------------------

\newcommand{\invoiceitem}[4]{

	\FPadd\gross{#2}{#3}\FPround\gross{\gross}{2}
	\global\let\variableA\gross % Set variable for use in table and other calculations
	
	%------------------------------------------------
	
	% Calculation of cumulative variables (total for invoice)
	%\FPeval{\beforetax}{round(\TotalA + \variableA, 2)} % Add the current before tax value to the previous total
	%\global\let\TotalA\beforetax % Set variable for display at the end and in other calculations
	
	% Output the table row
	#1 & % Item name
	\num{#2} & % Quantity
	\num{#3} & % Rate
	- &% Subtotal
	%\num{\variableA} &% Subtotal
	#4 \\
}

%-------------------------------------------------------
%           UNITPRICE
%--------------------------------------------------------

\newcommand{\unitprice}[3]{
	
	% Output the table row
	#1 & % Item name
	\num{#2} & % Quantity
	\num{#3} & % Rate
	&% Subtotal
	Rs / kWh \\
}



\newenvironment{billingdetails}
{
	\begin{tabular}{p{0.22\textwidth} R{0.22\textwidth} R{0.22\textwidth} R{0.22\textwidth}}
		\textbf{BILL DATE} & \textbf{BILL CYCLE} & \textbf{FROM DATE} & \textbf{TO DATE} \\
		\arrayrulecolor{rulecolour}\toprule[0.5pt]
		\billdate & \billcycle & \fromdate & \todate \\
	\end{tabular}
}


